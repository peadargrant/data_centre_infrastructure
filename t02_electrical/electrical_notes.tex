\chapter{Electrical basics}
\label{ch:electrical-basics}


\section{Physical quantities}

\autoref{tab:key-electrical-quantities} summarises key electrical quantities.

\begin{table}[htbp]
  \centering
  \begin{tabular}{l l l l}
    \toprule
    \textbf{Quantity} & \textbf{Symbol} & \textbf{Unit} & ~ \\
    \midrule
    Voltage & $V$ & Volt & \si{\volt} \\
    Current & $I$ & Ampere & \si{\ampere} \\
    Resistance & $R$ & Ohm & \si{\ohm} \\
    Power & $P$ & Watt & \si{\watt} \\
    \bottomrule
  \end{tabular}
  \caption{Key electrical quanities}
  \label{tab:key-electrical-quantities}
\end{table}

Note that voltage is a difference between \textit{two} points.
Often measured in any system with respect to a common earth / ground.

\section{Laws}

There are two key laws that relate basic electrical quantities, Ohm's Law and the Power Relation.

\subsection{Ohm's law}
\label{sec:ohms-law}
  
Voltage, current and resistance are related by Ohm's law, which can be written in terms of $V$, $I$ or $R$.
Re-arrange to calculate required quantity.
\begin{align}
  V & = R \cdot I \\
  \Rightarrow I & = \frac{V}{R} \\
  \Rightarrow R & = \frac{V}{I}      
\end{align}

\begin{example}{Ohm's law}{ohms-law-v}
  A \SI{7}{\ohm} resistance carries a current of \SI{2}{\ampere}.
  Determine the voltage across the component.
  \tcblower
  \begin{align}
    V & = 7 \times 2 \\
      & = \SI{14}{\volt}
  \end{align}
\end{example}

\begin{example}{Ohm's law with rearrangement}{ohms-law-r}
  When an electrical component is connected across a battery nominally supplying \SI{12}{\volt} a current of \SI{3}{\ampere} flows.
  Calculate the resistance of the component.
  \tcblower
  \begin{align}
    V & = R \cdot I \\
    \Rightarrow R & = \frac{V}{I} \\
      & = \frac{12}{3} \\
      & = \SI{4}{\ohm}
  \end{align}
\end{example}

\subsection{Power relation}
\label{sec:power-relation}

Power quantifies how much energy is converted from one form to another per unit time. Measured in Joule per second \si{\joule\per\second}, more commonly the Watt \si{\watt}.
Just as with Ohm's law, the power relation, \autoref{eq:power-P}, can be rearranged to give $V$ or $I$:
\begin{align}
  P & = V \cdot I \label{eq:power-P} \\
  \Rightarrow V & = \frac{P}{I} \label{eq:power-V} \\
  \Rightarrow I & = \frac{P}{V} \label{eq:power-I} 
\end{align}

\begin{example}{Power calculation}{power-V}
  A graphics card is supplied from the \SI{12}{\volt} power supply in a computer. The current flowing is measured at \SI{5}{\ampere}. Determine the power consumed by the graphics card.
  \tcblower
  \begin{align}
    P & = 12 \times 5 \\
      & = \SI{60}{\watt} 
  \end{align}
\end{example}

\begin{example}{Power calculation with rearrangement}{power-I}
  A computer power supply delivers \SI{6}{\watt} to a hard disk drive on the \SI{12}{\volt} line.
  Determine the current flowing in the cable.
  \tcblower
  \begin{align}
     I & = \frac{P}{V} = \frac{6}{12} \\
      & = \SI{0.5}{\ampere}
  \end{align}
\end{example}

\subsection{Combining}
\label{sec:combining}

Ohm's law and the power relation can be combined by substituting one quantity for another.

\section{Alternating current (AC)}
\label{sec:alternating-current}

Mains electricity is supplied in most parts of the world as alternating current (AC).
In Ireland and most of the EU, mains electricity is supplied at a \textit{nominal} voltage of \SI{230}{\volt} and frequency \SI{50}{\hertz}.
This means that the instantaneous voltage $v(t)$ varies sinusoidally with respect to time.
\begin{align}
  v(t) & = V_{\mbox{max}} \sin ( 2 \pi f t ) \label{eq:ac-instantaneous-voltage}
\end{align}
A single cycle of a generic AC waveform is shown in \autoref{fig:ac-waveform-properties}

\autoimage{ac_waveform_properties}{AC waveform properties}{ac-waveform-properties}

\subsection{Amplitude}
\label{sec:amplitude}

The \textbf{maximum voltage} $V_{\mbox{max}}$ of a sinusoid is the amplitude of the sine wave in both directions.
The most positive value is $V_{\mbox{max}}$ whilst the most negative value is $-V_{\mbox{max}}$.

\subsection{Peak-to-peak voltage}
\label{sec:peak-to-peak-voltage}

We can thus define the \textbf{peak-to-peak} amplitude as the difference between these two values:
\begin{align}
  V_{\mbox{PK-PK}} & = V_{\mbox{max}} - ( - V_{\mbox{max}} ) \\
                   & = 2 V_{\mbox{max}}                    
\end{align}

\begin{example}{Peak-to-peak to amplitude}{pkpk-amplitude}
  Calculate the amplitude of an AC waveform with a \SI{650}{\volt} peak-to-peak amplitude.
  \tcblower
  \begin{align}
    V_{\mbox{max}} & = \frac{V_{\mbox{PK-PK}}}{2} \\
                   & = \frac{650}{2} \\
                   & = \SI{325}{\volt} 
  \end{align}
\end{example}

\subsection{RMS Voltage}
\label{sec:rms-voltage}

The voltage in western Europe is a nominal \SI{230}{\volt}~RMS.
This is a root mean square value, which is equivalent to the heating power that the same DC voltage would deliver.
\begin{align}
  V_{\mbox{max}} & = \sqrt{2} V_{\mbox{RMS}} \\
  \Rightarrow V_{\mbox{RMS}} & = \frac{V_{\mbox{max}}}{\sqrt{2}}
\end{align}

\begin{example}{RMS to peak voltage}{rms-to-peak-voltage}
  Calculate the amplitude of a \SI{230}{\volt} RMS AC supply.
  \tcblower
  \begin{align}
    V_{\mbox{max}} & = \sqrt{2} V_{\mbox{RMS}} \\
                   & = \sqrt{2} \times 230 \\
                   & = \SI{325}{\volt}                    
  \end{align}
\end{example}

\begin{example}{Peak to RMS voltage}{peak-to-rms-voltage}
  Calculate the RMS voltage of an AC supply with an amplitude of \SI{100}{\volt}.
  \tcblower
  \begin{align}
    V_{\mbox{RMS}} & = \frac{V_{\mbox{max}}}{\sqrt{2}} \\
                   & = \frac{100}{\sqrt{2}} \\
                   & = \SI{70.7}{\volt} 
  \end{align}
\end{example}

\subsection{Frequency / Period}
A single cycle lasts for a period of time, $T$.
The period is directly related to the frequency:
\begin{align}
  T & = \frac{1}{f} \\
      \Rightarrow f & = \frac{1}{T}
\end{align}
\begin{example}{Frequency to period}{frequency-to-period}
  Calculate the period of a signal that repeats at \SI{20}{\hertz}.
  \tcblower
  \begin{align}
    T & = \frac{1}{20} \\
      & = \SI{0.05}{\second} 
  \end{align}
\end{example}
\begin{example}{Period to frequency}{period-to-frequency}
  Determine the frequency of a signal with a period of \SI{40}{\milli\second}.
  \tcblower
  \begin{align}
    f & = \frac{1}{\num{40e-3}} \\
      & = \SI{25}{\hertz}
  \end{align}
\end{example}


\section{Mains electricity}

Mains wiring in Ireland generally involves three conductors:
\begin{description}
\item[Live] carries a \SI{230}{\volt} RMS AC voltage.
\item[Neutral] provides the return path for current on the live conductor, and under normal conditions:
  \begin{enumerate}
  \item It will carry the same current in reverse as the live conductor.
  \item It's voltage (measured with respect to earth) will be zero.
  \end{enumerate}
\item[Earth] is connected to earth and bonded to metal casings.
\end{description}



\subsection{Circuit protection}

\begin{description}
\item[Fuses:] a piece of thin wire encased in a holder that is deliberatly designed to melt if the current exceeds the fuse rating.
\item[Circuit Breakers (MCB):] electromechanical devices that will trip when the current exceeds the circuit breaker's rating.
\item[Residual current device (RCD):] protect from electric shock by detecting any leakage of current to earth by comparing live and neutral currents.
  Trips if these differ by more than a set amount $\Delta I$, normally \SI{30}{\milli\ampere}.
  Other names: GFI, ELCB.
\item[Residual Current Breaker Overload (RCBO):] combined MCB and RCD functionality in one device.
\end{description}

\section{Computer power supplies}

Power Supply Units (PSUs) are used to convert the mains-supplied power into a form suitable for use in computers.

\subsection{Computer power requirements}

Computers require a variety of DC voltages, the most common being:
\begin{description}
\item[\SI{3.3}{\volt}] (orange)
\item[\SI{5}{\volt}] (red)
\item[\SI{12}{\volt}] (yellow).
\end{description}
These are normally supplied relative to a common ground (black).
\begin{description}
\item[Negative \SI{-12}{\volt}] (blue) is often available.
\end{description}
A PSU also permanently supplies \SI{5}{\volt} (purple) and will turn on when the green terminal is shorted to ground.

\subsection{Power supply tasks}

A computer's power supply unit (PSU) has two key jobs:
\begin{description}
\item[Rectify] the mains-supplied AC to a steady DC supply.
\item[Step down] the voltage (\SI{230}{\volt}) to the required level(s).
\end{description}
The precise methods and order that these tasks are performed in will vary, and are outside the scope of our discussion.

\subsection{Capacity}

A power supply will usually have a rated capacity:
\begin{itemize}
\item This will either be given in terms of power (watts) or in current (amps).
\item On a supply providing multiple voltages, there will usually be a limit on each rail as well as possibly an overall limit.
\end{itemize}

\section{Three-phase supply}

Mains power is generated and distributed in three-phase form, with 3 live conductors and one neutral conductor.
The sine wave is shifted by 120 degrees, or $\frac{2\pi}{3}$ radians in any phase relative to one of the two other phases.

\autoimage{3_phase_waveform}{3-phase waveform (Wikipedia)}{3-phase-waveform}

Let $v_n(t)$ be the voltage in phase $n$ of a three-phase supply.
We can use \autoref{eq:ac-instantaneous-voltage} for the first phase.
Phase~2 must lags phase 1 by 120 degrees.
Similarly, phase~3 leads phase~1 by 120 degrees.
\begin{align}
  v_1(t) & = V_{\mbox{max}} \sin ( 2 \pi f t ) \\
  v_2(t) & = V_{\mbox{max}} \sin \left ( 2 \pi f t - \frac{2\pi}{3} \right ) \\
  v_3(t) & = V_{\mbox{max}} \sin \left ( 2 \pi f t + \frac{2\pi}{3} \right )                      
\end{align}

\subsection{Line and phase voltages}

When dealing with three-phase power, we actually have two voltages to consider:
\begin{description}
\item[Phase voltage] is the voltage between \textit{any} phase and neutral.
\item[Line voltage] is the voltage measured between \textit{any} two phases.
\end{description}

The line and phase voltages are related mathematically by $\sqrt{3}$:
\begin{align}
  V_{\mbox{line}} & = \sqrt{3} \times V_{\mbox{phase}}\\
\Rightarrow  V_{\mbox{phase}} & = \frac{V_{\mbox{line}}}{\sqrt{3}}                     
\end{align}

\begin{example}{Line to phase voltage}{line-to-phase-voltage}
  A three phase power supply has a phase voltage of \SI{220}{\volt}.
  Calculate the line voltage.
  \tcblower
  \begin{align}
    V_{\mbox{line}} & = \sqrt{3} \times V_{\mbox{phase}} \\
                     & = \sqrt{3} \times 220  \\
                     & = \SI{381}{\volt}
  \end{align}
\end{example}

\begin{example}{Phase to line voltage}{phase-to-line-voltage}
  A three phase power supply has a line voltage of \SI{400}{\volt}.
  Calculate the phase voltage.
  \tcblower
  \begin{align}
    V_{\mbox{phase}} & = \frac{V_{\mbox{line}}}{\sqrt{3}} \\
                     & = \frac{400}{\sqrt{3}} \\
                     & = \SI{231}{\volt}
  \end{align}
\end{example}


