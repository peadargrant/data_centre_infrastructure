\documentclass{pgnotes}

\title{Cooling lab}

\begin{document}

\maketitle

Attempt the following problems.
You are also recommended to try and write scripts to solve these (in PowerShell, Python, Perl, R, or your other favourite scripting language!)

\begin{enumerate}

\item
  Convert 98F to C.

\item
  Convert 12C to F.

\item
  Convert 20C to K.

\item
  Convert 284K to C.
  
\item
  A small wiring closet has a \SI{500}{\watt} IT load within.
  Recommend the simplest suitable cooling solution to maintain a temperature of $\le \SI{25}{\celsius}$.

\item
  A data centre environment has a IT load of \SI{26}{\kilo\watt} in a closed space.
  Determine the cooling capacity required, stating key relevant assumptions that you make.
  
\item
  An air conditioning system consumes \SI{3.3}{\kilo\watt} to provide \SI{11.55}{\kilo\watt} of cooling.
  Determine the COP.
  
\item 
  An air conditioning system has a COP of 4.2.
  It consumes \SI{5.2}{\kilo\watt} of power.
  Calculate the amount of heat it is capable of removing.

\item
  An American Air Conditioning manufacturer advertises an Air Cooled CRAC with an EER of 13.5.
  Calculate the COP. % 3.96
  
\item
  Convert \SI{50000}{\BTU\per\hour} to \si{\kilo\watt}.

\item
  A cooling load of \SI{26}{\kilo\watt} is serviced by a cooling system with an EER of 12.
  Calculate the power consumption of this cooling system.

\item
  A small server room contains \SI{2}{\kilo\watt} of IT loads in a closed space.
  A fan is installed which consumes \SI{200}{\watt} at maximum speed.
  No other non-IT loads are located within the closed space.
  Determine the PUE. 

\item
  A server room contains \SI{5.8}{\kilo\watt} of IT loads in a closed space.
  A cooling system with a COP of 5.2 is proposed to be installed.
  No other non-IT loads are located within the closed space.
  Determine the PUE.
  
\item
  A data centre environment hosts \SI{16}{\kilo\watt} of IT loads in a closed space.
  The UPS units are located within the racks and add an additional \SI{25}{\percent} to the IT load at their current load.
  The cooling system has a COP of 4.5.
  Based on this information, determine the PUE.
  
\end{enumerate}


\end{document}

