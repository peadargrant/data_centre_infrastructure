\documentclass{pgnotes}

\title{Co-location}

\begin{document}

\maketitle

\section{Co-location}

Co-location is where a customer rents space from a so-called co-location facility, or co-lo.

\subsection{Renting}

Co-los rent space in various different units:
\begin{description}
\item[Rack spaces] in single/multiple units.
\item[Full cabinet] where the customer has use of a single cabinet in its entirety and can populate it as they wish.
\item[Cage] is a restricted area within the airspace of the data hall that is for a single customer's use.  They can normally populate the cabinets within it (supplied by the data centre) as they wish. 
\item[Suite] is where an entire data hall is provided to a single customer.
\end{description}

\subsection{Networking}

Within a co-lo environment, the customer is solely responsible for their own networking.
This means that the co-lo provider \textit{does not} supply IP addresses or outbound connectivity as standard.
Customers must have the appropriate networking cables and equipment to adequately connect their own equipment together.

\subsection{Cross-connects}

Customers may sometimes have equipment in multiple cabinets, or may wish to connect directly to other data centre customers.
They can do this by ordering a cross-connect from the co-lo provider.
Depending on the provider, there may be a charge for this service.

\subsection{Carrier connection}

Most links to locations away from the data centre will be made via telecommunication carriers.

Whilst some data centres historically have restricted the carriers a customer can use, indeed many have been operated by telecommunications carriers themselves, most are said to be carrier-neutral.

Carriers normally terminate their connections in a so-called meet-me room (MMR).
The meet-me room allows patching through cross-connects to the customer's equipment in the data hall. 

It is the responsibility of the customer to order the WAN service from the carrier, and to order the appropriate cross connect from the co-lo provider.

\end{document}

