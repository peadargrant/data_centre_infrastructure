\documentclass{pgnotes}

\title{Generators}

\begin{document}

\maketitle

\section{Load types}

\begin{description}
\item[Critical] loads cannot lose power, even momentarily.
\item[Essential] loads can tolerate a brief power interruption, but will need to be operational for continued running of the data centre.
\item[Non-essential] loads are discretionary and can be done without. 
\end{description}

The classification of any particular load will depend on the business needs.
The same load might be classed as critical, essential or non-essential in different contexts.

\section{Standby generators}

A UPS normally protects critical loads for a short period of time.
For longer power outages, a separate generator is required.

Standby generators are usually supplied as a package unit, containing:
\begin{description}
\item[Prime-mover] normally a reciprocating diesel engine.
\item[Alternator] to convert mechanical energy into electrical energy.
\item[Controls] including a governor to control engine speed, determining frequency, and an automatic voltage regulator to control the excitation in the generator's windings, determining output voltage.
\end{description}

\autoimage{generator_schematic}{Diesel generator schematic}{generator-schematic}

\section{Power system including Generator}

The generator normally powers the essential and (via the UPS) the critical loads, \autoref{fig:load-types}.

\autoimage{load_types}{Standby generator arrangement}{load-types}

\section{Transfer sequence}

The transfer is usually managed by an automatic transfer switch.
This is a ``break-before-make'' switch that disconnects the power from the mains and transfers it to the generator.
The transfer switch is normally interlocked with the generator's control system so that the generator starts on a loss of power.

\subsection{On loss of mains power}

\begin{enumerate}
\item Power is lost in the mains (i.e. voltage goes out of range). This causes the automatic transfer switch (ATS) to start a short time delay.  If power returns, the sequence terminates, otherwise the ATS signals the generator to start.
\item The generator will take some time (1-2 minutes) to:
  \begin{enumerate}
  \item Start the prime mover (engine), possibly needing multiple re-trys.
  \item Engine accelerates to correct speed to produce \SI{50}{\hertz}.
  \item Excitation and automatic voltage regulation to cause correct voltage to appear \SI{230}{\volt} on the output.
  \end{enumerate}
\item ATS will break the connection to the mains and make it instead to the generator.\label{item:ats-switch-to-generator}
\item Generator maintains frequency at \SI{50}{\hertz} using governor to control engine throttle, and voltage at \SI{230}{\volt} using automatic voltage regulator that controls the excitation current.
\end{enumerate}

\subsection{On return of mains power}

\begin{enumerate}
\item ATS senses return of mains power. Initiates a time delay before continuining to ensure mains power is stable.
\item ATS transfers loads back to mains. Generator continues to run for a predetermined cooldown period.
  \begin{itemize}
  \item If mains power is again lost during this period, can switch back to generator almost immediately.
  \end{itemize}
\item Generator is stopped once cooldown period has elapsed.
  Ready for cycle to resume again.
\end{enumerate}

\section{UPS compatibility}

Generators and UPS systems have a complex relationship.
Although they synergise well in principle, they can both cause problems for each other in practice.

\begin{description}
\item[Standby \& line-interactive] UPS will require the generator to maintain a very tight tolerence on voltage and frequency.
  Frequent transfers to battery possible when $V$ and $f$ deviate due to load switching.
\item[Double-conversion (AC-DC-AC)] UPS will generally work best, as it is tolerent to incoming frequency/voltage variations. \textit{Double-conversion UPS systems can cause major problems though for undersized generators because their rectifiers can generate harmonics on the power input.}
\end{description}

In general, the less dominant the UPS (and its downstream load) is as a fraction of the total load, the more successful the UPS-generator pairing will be.
More generally, a generator will be more successful powering a large number of smaller loads than a smaller number of large loads.

\section{Generator Sizing}

Generators need to be sized taking into account:

\begin{itemize}
\item The critical load attached to the UPS.
\item The battery recharging demand of the UPS.
\item Inherent inefficiencies in the UPS.
\item The essential loads (such as cooling).
\item Large starting currents and harmonics when loads (including the UPS rectifier) are connected, particularly when starting up.
\item Expansion capacity (since a generator is a large capital investment).
\end{itemize}

This means that substantial buffer / correction factors are often applied to ensure a generator is adequately sized.

\section{Responsibility}

Generators are often not managed by IT personnel, but rather facilities management.
IT will need to communicate with facilities managament both to maintain a good work relationship and to exchange data, including the setup of automated data feeds.


\section{Siting}

Many generators are sited outside, but some are sited inside, depending on business needs and space planning.
Consideration must be given to:
\begin{description}
\item[Ease of interconnection] to power system.
\item[Noise] generated in use inside and outside building.
\item[Fuel storage and supply] using inbuilt and/or external tanks, gravity or pumped supply.
\item[Exhaust system] to ensure safety of building fabric and its occupants.
\item[Accessibility] for servicing and routine maintenance and testing.
\item[Cooling] of the generator itself during operation.  If inside, ventilation/cooling will be needed.
\item[Security] since sabotage could ``take out'' a data centre!
\end{description}

\section{Routine maintenance}

\begin{description}
\item[Starting battery] needs to be kept charged and checked on schedule.
\item[Engine-block heating] (usually electric) should be considered in cold climates.
\item[Generator must be tested] regularly, preferably on load.
\end{description}

\end{document}






